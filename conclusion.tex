\section{Conclusion}
Our approach significantly simplifies the optimal usage of the FETI solvers in ESPRESO library 
and can provided significant savings in orders of magnitude for both runtime as well as energy consumption for inexperienced users. 
The final model as implemented requires only following input parameters provided by user: 

\begin{itemize}
\item total problem size,
\item selected FETI method, 
\item selected preconditioner, 
\item estimated number of CG iterations (used to estimate the significance of the preprocessing routines).
\end{itemize}

while it returns the estimated solver run-time with the corresponding optimal 
decomposition, parallelization settings of the solver.

The number of CG iteration can not be predicted by our model so we left it as the user input.  
User can simply perform a single test-runs,  which will give them the approximate information 
about the number of iterations. 

The estimated run-time was predicted with deviation lesser than 1\,s, given the 
measured minimal time, as can be seen in the Tab. \ref{tab:finalModelTest}. 
This is an acceptable precision, as with the least suitable settings the
run-time can be more than 20 times (i.e. more than 50\,s in this case) slower  
as the the optimal one.
